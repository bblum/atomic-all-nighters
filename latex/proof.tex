\documentclass{article}
\usepackage{amsmath,amsthm,amssymb,fullpage,yfonts,graphicx,proof,appendix,hyperref,mdwlist,wasysym}
\usepackage{stmaryrd}
\usepackage{mathpartir}
\usepackage{upgreek}
\usepackage{fge}
\usepackage{epsfig}
\usepackage{caption}
\usepackage{subcaption}
\usepackage[bottom]{footmisc}
\usepackage{bera}
\usepackage[T1]{fontenc}

\begin{document}
\captionsetup{width=.75\textwidth,font=small,labelfont=bf}
\title{{\bf Atomic All-Nighters: \\ Static Context Checking in Kernel Codebases} \\ {\em Correctness Proof}}
\author{Ben Blum (\texttt{bblum@cs.cmu.edu})}
\maketitle

\newcommand\true{\;\textit{true}}
\newcommand\false{\;\textit{false}}

\newcommand\alpher\alpha
\newcommand\beter\beta
\newcommand\gammer\gamma
\newcommand\delter\delta
\newcommand\zeter\zeta
\newcommand\Sigmer\Sigma

\newcommand\NN{\mathbb{N}}
\newcommand\QQ{\mathbb{Q}}
\newcommand\RR{\mathbb{R}}
\newcommand\ZZ{\mathbb{Z}}

\newcommand\Infinity{ {\sf Infinity}}
\newcommand{\Nested}[1]{ {\sf Nested}~ #1}
\newcommand{\IncDec}[1]{ {\sf IncDec}~ #1}
\newcommand\Enable{ {\sf Enable}}
\newcommand\Disable{ {\sf Disable}}
\newcommand\inc{ {\sf inc}}
\newcommand\dec{ {\sf dec}}

\newcommand\sleep{{\sf sleep}}
\newcommand\unit{{1}}

\begin{abstract}
This is a correctness proof for the formal representation of the Atomic All-Nighters ``type-check''.
% The formalism is for ``straight-line`` programs only (i.e., programs with no flow control), though variables and function pointers are included. Programs are represented as a set of functions that operate as a stack machine.
The formalism is for simply-typed lambda calculus programs, so invariance and mutable storage locations are not included. Higher-order functions, contravariance, and covariance are featured nevertheless.
We aim to prove that, if a program passes the check, it cannot invoke the \sleep~primitive while the context is nonzero.
\end{abstract}


\section{Definitions}

The grammar consists of contexts ($C$), effects ($E$), rules ($R$), annotations ($A$), types ($\tau$),
% expressions ($e$), and programs ($p$).
and expressions ($e$).

\newcommand\expr{\ensuremath{\mathsf{e}}}
\newcommand\stmt{\ensuremath{\mathsf{st}}}
\newcommand\decl{\ensuremath{\mathsf{decl}}}
\newcommand\fn{\ensuremath{\mathsf{fn}}}
\newcommand\lv{\ensuremath{\mathsf{lv}}}
\newcommand\prog{\ensuremath{\mathcal{P}}}
\newcommand\fnname{\ensuremath{f}}
\newcommand\varname{\ensuremath{x}}
\newcommand\addr{\ensuremath{\&}}
\newcommand\rep[1]{\ensuremath{\overline{#1}}}
\newcommand\return{\ensuremath{\mathsf{return}}}
\newcommand\ctx[2]{\ensuremath{#1@#2}}

%\begin{figure}[h]
%	\begin{subfigure}[b]{0.5\textwidth}
%	\begin{eqnarray*}
%		C & ::= & \Nested{X} ~|~ {\Infinity} \\
%		R & ::= & C \\
%		E & ::= & \IncDec{X} ~|~ {\Enable} ~|~ {\Disable} \\
%		A & ::= & (R,E) \\
%		\tau & ::= & \unit ~|~ \rep{\tau} \rightarrow \tau \{A\} ~|~ *\tau \\
%	\end{eqnarray*}
%	\caption{The type language.}
%	\end{subfigure}
%	\begin{subfigure}[b]{0.5\textwidth}
%	\begin{eqnarray*}
%		\expr & ::= & () ~|~ \fnname ~|~ \fnname(\rep{\varname}) ~|~ \varname ~|~ \addr \varname ~|~ *\varname \\
%		\lv   & ::= & \varname ~|~ *\varname \\
%		\stmt & ::= & \lv = \expr \\
%		\decl & ::= & \tau~\varname \\
%		\fn   & ::= & \tau~\fnname(\rep{\decl}) \{ \rep{\decl;}~\rep{\stmt;}~\return~\expr \} \\
%		\prog & ::= & \rep{\fn}
%		% e & ::= & x ~|~ \lambda x:\tau . e' \{A\} ~|~ {\inc} ~|~ {\dec} ~|~ {\sleep}
%	\end{eqnarray*}
%	\caption{The term language.}
%	\end{subfigure}
%\end{figure}
\begin{eqnarray*}
	C & ::= & \Nested{X} ~|~ {\Infinity} \\
	R & ::= & C \\
	E & ::= & \IncDec{X} ~|~ {\Enable} ~|~ {\Disable} \\
	A & ::= & (R,E) \\
	\tau & ::= & \unit ~|~ \rep{\tau} \rightarrow \tau \{A\} ~|~ *\tau \\
	e & ::= & x ~|~ \lambda x:\tau . e' \{A\} ~|~ e_1 e_2 ~|~ {\inc} ~|~ {\dec} ~|~ {\sleep} ~|~ () \\
	\prog ::= \ctx{e}{C} \\
\end{eqnarray*}

\begin{itemize}
	\item A context $C$ represents the current state of the program. It can roughly be thought of as indicating the number of spinlocks that the program is currently holding.
	\item A rule $R$ represents the ``most restrictive'' context that a function is allowed to be called in. For a function that might sleep, the rule must be $\Nested{0}$. For the rest of the proof we will treat rules and contexts as interchangeable.
	\item An effect $E$ represents how the current context of a program is changed when it calls a function (i.e., the aggregate effect between invocation and return).
	\item An annotation $A$ characterises the complete behaviour of a function (with respect to AAN).
	\item A type $\tau$ is associated with each function and variable (function pointer) in the program. It can be either unit, a function type with an associated annotation, or a pointer (representing mutable memory).
	%\item A program \prog~is a list of functions.
	%\item A function \fn~consists of a return type, a name, a list of parameters, and an execution block consisting of a list of variable declarations, statements, and the return of an expression.
	%\item A declarator \decl~introduces a variable of a given type and name into the current block's scope.
	%\item A statement \stmt~assigns the value of an expression to a storage location.
	%\item An lvalue \lv~specifies a storage location: a variable or the dereference of a variable.
	%\item An expression \expr~can be either unit, the name of a function (creating a function pointer), a function call, the name of a variable, the address of a variable, or the dereference of a variable.
	\item An expression \expr~is the familiar lambda calculus expression with a few new primitives. Lambda terms are explicitly annotated, and there are additional terms for unit, increment-context, decrement-context, and sleep. For convenience we will also use the notation $e_1;e_2$ as sugar for $(\lambda _:\tau_1. e_2) e_1$ (where $\tau_1$ is whatever $e_1$ typechecks as). (The operational semantics are, of course, call-by-value.)
\end{itemize}

\newcommand\eff[3]{\ensuremath{#1 \lightning #2 \rightsquigarrow #3}}
\newcommand\subctx\preccurlyeq
\newcommand\subann\unlhd
\newcommand\effecteq\equiv
\newcommand\subtype{<:}
\newcommand\callsite[2]{\ensuremath{#1 @ #2 \checkmark}}
\newcommand\synth[2]{\ensuremath{#1 \fgerightarrow #2}}

\section{Judgements and Rules}

\begin{figure}[h]
	\begin{subfigure}[b]{0.5\textwidth}
	\begin{tabular}{rl}
		Context ordering & $C_1 \subctx C_2$ \\
		Effect equality & $E_1 \effecteq E_2 $ \\
		Annotation ordering & $A_1 \subann A_2 $ \\
		Subtyping & $\tau_1 \subtype \tau_2 $ \\
		Call-site legality & $\callsite{A}{C} $ \\
		Effect synthesis & $\Gamma \vdash \synth{e}{E}$ \\
		Annotation checking & $\Gamma \vdash e : \tau \{A\}$ \\
	\end{tabular}
	\caption{Checking judgements}
	\end{subfigure}
	\begin{subfigure}[b]{0.5\textwidth}
	\begin{tabular}{rl}
		Effecting contexts & $\eff{E}{C_1}{C_2} $ \\
		Stepping & $\ctx{e_1}{C_1} \mapsto \ctx{e_2}{C_2} $ \\
	\end{tabular}
	\caption{Evaluation judgements}
	\end{subfigure}
	\caption{Summary of judgements.}
	\label{fig:judge}
\end{figure}

\begin{figure}[h]
	% subcontext
	\fbox{$C_1 \subctx C_2$}
	\begin{mathpar}
	\infer[\subctx I]{\Infinity \subctx C}{} \qquad
	\infer[\subctx N]{\Nested{X} \subctx \Nested{Y}}{X \ge Y} \qquad
	\end{mathpar}
	% effect eq
	\fbox{$E_1 \effecteq E_2$}
	\begin{mathpar}
	\infer[\effecteq I]{\IncDec{X} \effecteq \IncDec{X}}{} \qquad
	\infer[\effecteq E]{\Enable \effecteq \Enable}{} \qquad
	\infer[\effecteq D]{\Disable \effecteq \Disable}{} \qquad
	\end{mathpar}
	% subann
	\fbox{$A_1 \subann A_2$}
	\begin{mathpar}
	\infer[\subann]{(R_1,E_1) \subann (R_2,E_2)}{R_1 \subctx R_2 \quad E_1 \effecteq E_2} \qquad
	\end{mathpar}
	TODO
	\caption{Static checking rules.}
	\label{fig:checking}
\end{figure}

Figure \ref{fig:judge} summarises the static and dynamic judgements.
Figure \ref{fig:checking} depicts the checking rules in detail.

\section{Preservation}

The preservation statement for System AAN is:

\begin{center}
	If $\vdash e_1 : \tau\{A_1\}$ and $e_1\{A_1\}@C_1 \mapsto e_2\{A_2\}@C_2$, then $\vdash e_2 : \tau\{A_2\}$.
\end{center}

Intuitively, if the rules of the annotation-checking system verify an annotated program as legal, then it will never invoke the {\sf sleep} primitive when the context value is greater than zero.
\footnote{Notably, to make this statement provable, we need to allow for the counter in the context to go negative (and for sleeping to be allowed in any non-positive context). This prevents us from detecting the possible illegal behaviour of trying to re-enable preemption when it is already on, but this does not impact our ability to find the actual atomic-sleep bug.}

The proof of this is left to future work.
\footnote{We are not sure if it is actually provable.}


\end{document}
